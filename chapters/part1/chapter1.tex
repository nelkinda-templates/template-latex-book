\chapter{Introduction to C}
\epigraph{``C is quirky, flawed, and an enormous success.''}{\em Dennis M. Ritchie\index{Ritchie, Dennis M.}\em}

The C Programming Language, short ``C'', was developed by \index{Ritchie, Dennis M.}Dennis M. Ritchie.

C first appeared in 1972, which makes it quite an old language by today's standards.
Despite its age, C still is in widespread use, and will remain so for the foreseeable future.

The major fields of application for C include:
\begin{itemize}
\item \Gls{firmware}
\item Operating system kernels
\item Drivers
\item Basic operating system utilities and programs
\item Core of virtual machines, runtime environments, and interpreters
\end{itemize}

C was made popular with the book ``The C Programming Language''\cite{kernighanAndRitchie} by \index{Kernighan, Brian}Brian Kernighan and \index{Ritchie, Dennis M.}Dennis M. Ritchie.
This book is often referred to under the name ``The Kernighan \& Ritchie''.
It has made it famous and popular to introduce a programming language with a very simple first listing that displays the message ``Hello, world!'' on the screen.

\lstinputlisting[caption={\index{Hello, World!C}Hello, World! in C}, label={lst:hello.c}]{src/hello.c}

The C Programming Language is the language in which Linux is written.
It is also the language of much of the \acrshort{acronym:GNU} software.
